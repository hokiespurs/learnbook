\section{Ordinary Least Squares (OLS)}
\subsection{Theory}
Ordinary Least Squares solves a linear, overconstrained system of equations by minimizing the squared residuals of each observation equation.
\[
min \sum_{i}^{n} v_i^2
\]

\subsection{Assumptions}
\begin{itemize}
	\item No Outliers/Blunders OLS is not robust to outliers (consider RANSAC/Robust Weighting if outliers)
	\item System of equations is linear (eg. derivative wrt each unknown is not a function of any of the unknowns)
	\item System is over-contrained (eg. Number of Observation Equations > Number of Unknowns)
	\item Error only in dependent variable (eg. mx+b = y + v $\rightarrow$ error only in y dimension)
\end{itemize}
\subsection{Equations}
\[
AX=L+V 
\]
\[
m = \text{number of observations} \hspace{1cm} 
n = \text{number of unknowns}
\]
\[
dof = \text{degrees of freedom (\# of redundant observations)} = m-n
\]
\[
A = \begin{bmatrix}
a_{11} & a_{12} & \dots & a_{1n} \\
a_{21} & a_{22} & \dots & a_{1n} \\
\vdots & \vdots & \vdots& \vdots \\
a_{m1} & a_{m2} & \dots & a_{mn} \\
\end{bmatrix}
\hspace{1cm}
X = 
\begin{bmatrix}
x_1 \\ x_2 \\ \vdots \\ x_n
\end{bmatrix}
\hspace{1cm}
L = 
\begin{bmatrix}
l_1 \\ l_2 \\ \vdots \\ l_m
\end{bmatrix}
\hspace{1cm}
V = 
\begin{bmatrix}
v_1 \\ v_2 \\ \vdots \\ v_m
\end{bmatrix}
\]
\begin{align*}
\text{Unknowns} &= \hat{X} = inv(A^TA)A^TL\\
\text{Residuals} &= V = AX - L\\
\text{Reference Variance} &= S_0^2 = \dfrac{V'V}{dof} \\
\text{Cofactor Matrix} &= Q_{xx} = inv(A^TA) \\
\text{Covariance Matrix of Unkowns} &= \Sigma_{xx} = S_0^2 Q_{xx} \\
\text{Covariance Matrix of Observations} &= \Sigma_{\hat{l}\hat{l}} = A \Sigma_{xx} A^T \\
\text{Standard Deviation of Solved Unknowns} &= \sigma_{\hat{X}} = \sqrt{diag(\Sigma_{xx})} \\
\text{Predicted L} &= \hat{L} = AX \\
\text{R$^2$ (model skill)} &= \dfrac{var(\hat{L})}{var(L)} \\
\text{RMSE } &= \sqrt{\dfrac{VV^T}{m}} \\
\end{align*}
\clearpage
\subsection{Sample Problem}
Given the Points: 
\[
x = [0,1,2,3,4] \hspace{1cm} y = [0,1,7,13,24]
\]
Fit a parabola given the observation equation:
\[
y = a x^2 +b x + c
\]
Note that the observation equation is linear. The x$^2$ term is a constant once observation values are substituted.
\[
A = \begin{bmatrix}
x_1^2 & x_1 & 1 \\
x_2^2 & x_2 & 1 \\
x_3^2 & x_3 & 1 \\
x_4^2 & x_4 & 1 \\
x_5^2 & x_5 & 1 \\
\end{bmatrix} =
\begin{bmatrix}
0 & 0 & 1 \\
1 & 1 & 1 \\
4 & 2 & 1 \\
9 & 3 & 1 \\
16 & 4 & 1 \\
\end{bmatrix}
\hspace{1cm}
X = 
\begin{bmatrix}
a \\ b \\ c 
\end{bmatrix}
\hspace{1cm}
L =
\begin{bmatrix}
y_1 \\ y_2 \\ y_3 \\ y_4 \\ y_5
\end{bmatrix} = 
\begin{bmatrix}
0 \\ 1 \\ 7 \\ 13 \\ 24
\end{bmatrix}
\]
Use the Equations and solve:
\begin{table}[htbp]
	\centering
	\begin{tabular}{|c|c|c|}
		\toprule
		$dof = 2$
		&
		$\hat{X} = 
		\begin{bmatrix}
		1.43\\0.29\\-0.14
		\end{bmatrix}$
		&
		$V = 
		\begin{bmatrix}
		-0.14\\0.57\\-0.86\\0.57\\-0.14
		\end{bmatrix}$
		\\	\midrule
		$Q = \begin{bmatrix}
		0.07 & -0.29 & 0.14 \\ 
		-0.29 & 1.24 & -0.77 \\ 
		0.14 & -0.77 & 0.89 \\ 
		\end{bmatrix}$
		&
		$S_0^2 = 0.71$
		&
		$\hat{L} = 
		\begin{bmatrix}
		-0.14\\1.57\\6.14\\13.57\\23.86
		\end{bmatrix}$
		\\	\midrule
		$\Sigma_{xx} = \begin{bmatrix}
		0.05 & -0.20 & 0.10 \\ 
		-0.20 & 0.89 & -0.55 \\ 
		0.10 & -0.55 & 0.63 \\ 
		\end{bmatrix}$
		&
		$\sigma_{\hat{X}}^2 = 
		\begin{bmatrix}
		0.05\\0.89\\0.63 
		\end{bmatrix}$
		&
		$R^2 = 0.9963$
		\\ \midrule
		$\Sigma_{\hat{l}\hat{l}} = \begin{bmatrix}
		0.05 & -0.20 & 0.10 \\ 
		-0.20 & 0.89 & -0.55 \\ 
		0.10 & -0.55 & 0.63 \\ 
		\end{bmatrix}$
		& & \\
		\bottomrule
	\end{tabular}%
\end{table}%
\begin{figure}[H]
	\centering
	\includegraphics[height = 3in]{OLSexample.png}
\end{figure}
\clearpage

\subsection{Example Matlab Code}
\todo{check this code (so2 squared when it shouldn't)}
\todo{show code using matlab lscov}
\lstinputlisting[
label      = {alg:exampleOLS},
caption    = {exampleOLS.m},
style      = Matlab-editor,
basicstyle = \mlttfamily,
firstline  = 1,
lastline   = 17,
firstnumber= 1
]{exampleOLS.m}

\lstinputlisting[
label      = {alg:exampleOLSplot},
caption    = {exampleOLS.m (plotting Code)},
style      = Matlab-editor,
basicstyle = \mlttfamily,
firstline  = 19,
lastline   = 45,
firstnumber= 19
]{exampleOLS.m}