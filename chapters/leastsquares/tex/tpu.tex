\section{Error Propagation (GLOPOV/SLOPOV)}
Unknown values (Z) are often calculated from measurements (x), which contain uncertainty ($\Sigma_x$).  

For example:

\begin{align*}
Z_1 &= F_1(x_1,x_2,\dots,x_n) \\
Z_2 &= F_2(x_1,x_2,\dots,x_n) \\
&\vdots \\
Z_m &= F_m(x_1,x_2,\dots,x_n) \\
\end{align*}

With the Covariance Matrix:
\[
\begin{bmatrix}[1.5]
\sigma^2_{x_{1}} & \sigma_{x_{1}x_{2}} & \hdots & \sigma_{x_{1}x_{n}} \\
\sigma_{x_{1}x_{2}} & \sigma^2_{x_{2}} & \hdots & \sigma_{x_{2}x_{n}} \\
\vdots & \vdots & \ddots & \vdots \\
\sigma_{x_{1}x_{n}} & \sigma_{x_{2}x_{n}} & \hdots & \sigma^2_{x_{n}} \\
\end{bmatrix}
\]

The propagation of uncertainty is governed by the \textit{general law of propagation of variances} (GLOPOV).
\subsection{GLOPOV}
The equation for GLOPOV is given as:
\begin{align*}
\Sigma_{ZZ} &= 
A\Sigma A^T \\
\Sigma_{ZZ} &= 
\begin{bmatrix}[1.5]
\ddx{Z_1}{x_1} & \ddx{Z_1}{x_2} & \hdots & \ddx{Z_1}{x_n} \\
\ddx{Z_2}{x_1} & \ddx{Z_2}{x_2} & \hdots & \ddx{Z_2}{x_n} \\
\vdots & \vdots & \ddots & \vdots \\
\ddx{Z_m}{x_1} & \ddx{Z_m}{x_2} & \hdots & \ddx{Z_m}{x_n} \\
\end{bmatrix}
\times
\begin{bmatrix}[1.5]
\sigma^2_{x_{1}} & \sigma_{x_{1}x_{2}} & \hdots & \sigma_{x_{1}x_{n}} \\
\sigma_{x_{1}x_{2}} & \sigma^2_{x_{2}} & \hdots & \sigma_{x_{2}x_{n}} \\
\vdots & \vdots & \ddots & \vdots \\
\sigma_{x_{1}x_{n}} & \sigma_{x_{2}x_{n}} & \hdots & \sigma^2_{x_{n}} \\
\end{bmatrix}
\times
\begin{bmatrix}[1.5]
\ddx{Z_1}{x_1} & \ddx{Z_2}{x_1} & \hdots & \ddx{Z_m}{x_1} \\
\ddx{Z_1}{x_2} & \ddx{Z_2}{x_2} & \hdots & \ddx{Z_m}{x_2} \\
\vdots & \vdots & \ddots & \vdots \\
\ddx{Z_1}{x_n} & \ddx{Z_2}{x_n} & \hdots & \ddx{Z_m}{x_n} \\
\end{bmatrix}
\end{align*}
For Example:

The slope and y-intercept of a linear line are calculated using least squares.  The values are:
\[
m = 1.25 \hspace{1cm} b = 0.3 \hspace{1cm} \Sigma_x = 
\begin{bmatrix}[1.5]
0.2 & -1 \\
-1 & 10 \\
\end{bmatrix}
\]
Calculate the expected value and uncertainty of points:
\[
x_1 = 3.0\pm 0 \hspace{1cm} x_2 = 5.0 \pm 0.2 
\]
Using GLOPOV:
\begin{align*}
Z_1: \hspace{1cm} y &= mx_1+b = (1.25)(3.0) + (0.3) \\
Z_2: \hspace{1cm} y &= mx_2+b = (1.25)(5.0) + (0.3) \\
\end{align*}
\[
\Sigma = 
\begin{bmatrix}[1.5]
\sigma^2_{m} & \sigma_{mb}  & 0 & 0\\
\sigma_{mb} & \sigma^2_{b}  & 0 & 0\\
0 & 0  & \sigma^2_{x_1} & 0 \\
0 & 0  & 0 & \sigma^2_{x_2} \\
\end{bmatrix}
=
\begin{bmatrix}[1.5]
0.2 & -1  & 0 & 0\\
-1 & 10  & 0 & 0\\
0 & 0  & 0 & 0 \\
0 & 0  & 0 & 0.2 \\
\end{bmatrix}
\]
\[
A = 
\begin{bmatrix}[1.5]
\ddx{Z_1}{m} & \ddx{Z_1}{b} & \ddx{Z_1}{x_1} & \ddx{Z_1}{x_2} \\
\ddx{Z_2}{m} & \ddx{Z_2}{b} & \ddx{Z_2}{x_1} & \ddx{Z_2}{x_2} \\
\end{bmatrix}
= 
\begin{bmatrix}[1.5]
x_1 & 1 & m & 0 \\
x_2 & 1 & 0 & m \\
\end{bmatrix}
=
\begin{bmatrix}[1.5]
3.0 & 1 & 1.25 & 0 \\
5.0 & 1 & 0 & 1.25 \\
\end{bmatrix}
\]
\[
\Sigma_{ZZ} = A\Sigma A^T = 
\begin{bmatrix}[1.5]
3.0 & 1 & 1.25 & 0 \\
5.0 & 1 & 0 & 1.25 \\
\end{bmatrix}
\begin{bmatrix}[1.5]
0.2 & -1  & 0 & 0\\
-1 & 10  & 0 & 0\\
0 & 0  & 0 & 0 \\
0 & 0  & 0 & 0.2 \\
\end{bmatrix}
\begin{bmatrix}[1.5]
3.0   & 5.0  \\
1   & 1  \\
1.25 & 0 \\
0 & 1.25 \\
\end{bmatrix}
= 
\begin{bmatrix}
5.80&5.00\\
5.00&5.31\\
\end{bmatrix}
\]

\subsection{SLOPOV}
With \textbf{no covariances}, and only \textbf{one function Z}, GLOPOV can be simplified by the \textit{special law of propagation of variances} SLOPOV:
\[
\Sigma = 
\begin{bmatrix}
\sigma^2_{x_{1}} & 0 & \hdots & 0 \\
0 & \sigma^2_{x_{2}} & \hdots & 0 \\
\vdots & \vdots & \ddots & \vdots \\
0 & 0 & \hdots & \sigma^2_{x_{n}} \\
\end{bmatrix}
\]
\[
\sigma_{Z_1} = \sqrt{(\ddx{Z_1}{x_1} \sigma_{x_1})^2+(\ddx{Z_1}{x_2} \sigma_{x_2})^2+\dots+(\ddx{Z_1}{x_n} \sigma_{x_n})^2}
\]

\vspace{0.5cm}
\noindent
So for example, if the sides of a fish tank are measured as:
\begin{align*}
L &= 10.1 cm \pm 0.25cm \\
W &= 4.7 cm  \pm 0.03cm \\
H &= 6.3 cm  \pm 0.10cm \\
\end{align*}

Calculate the volume of the fish tank using the equation:
\begin{align*}
V &= LWH \\
V &= (10.1cm)(4.7cm)(6.3cm) \\
V &= 299.06 cm^3
\end{align*}

Solve Estimated $\sigma$ using SLOPOV:
\begin{align*}
\sigma_{V} &= \sqrt{(\ddx{V}{L} \sigma_{L})^2+(\ddx{V}{W} \sigma_{W})^2+(\ddx{V}{H} \sigma_{H})^2} \\
\sigma_{V} &= \sqrt{(WH (0.25cm))^2+(LH (0.03cm))^2+(LW (0.10cm))^2} \\
\sigma_{V} &= \sqrt{((4.7cm)(6.3cm)(0.25cm))^2+((10.1cm)(6.3cm) (0.03cm))^2+((10.1cm)(4.7cm) (0.10cm))^2} \\
\sigma_V &= 9.00 cm^3
\end{align*}

Answer:

\[
V = 299.06 cm^3 \pm 9.00 cm^3
\]